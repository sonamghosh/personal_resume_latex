% resume.tex
%
% (c) 2002 Matthew Boedicker <mboedick@mboedick.org> (original author) http://mboedick.org
% (c) 2003-2007 David J. Grant <davidgrant-at-gmail.com> http://www.davidgrant.ca
% (c) 2007-2014 Todd C. Miller <Todd.Miller@sudo.ws> http://www.sudo.ws/todd
%
% This work is licensed under the Creative Commons Attribution-ShareAlike 3.0 Unported License. To view a copy of this license, visit http://creativecommons.org/licenses/by-sa/3.0/ or send a letter to Creative Commons, 171 Second Street, Suite 300, San Francisco, California, 94105, USA.

%\documentclass[letterpaper,10pt]{article}
\documentclass[letterpaper, 8pt]{extarticle}
%-----------------------------------------------------------
\usepackage[empty]{fullpage}
\usepackage{color}
\usepackage{fontawesome}
\usepackage{enumitem}
\usepackage{comment}
\definecolor{mygrey}{gray}{0.80}
\raggedbottom
\raggedright
\setlength{\tabcolsep}{0in}
%\setlength{\topmargin}{-0.25in}  % change this later
% Adjust margins to 0.5in on all sides
\addtolength{\oddsidemargin}{-0.5in}
\addtolength{\evensidemargin}{-0.5in}
\addtolength{\textwidth}{1.0in}
\addtolength{\topmargin}{-0.5in}
\addtolength{\textheight}{1.0in}
%-----------------------------------------------------------
%Custom commands
\newcommand{\resitem}[1]{\item #1 \vspace{-2pt}}
\newcommand{\resheading}[1]{{\large \colorbox{mygrey}{\begin{minipage}{0.99\textwidth}{\textbf{#1 \vphantom{p\^{E}}}}\end{minipage}}}}

\newcommand{\ressubheading}[4]{
\begin{tabular*}{7.40in}{l@{\extracolsep{\fill}}r}
		\textbf{#1} & #2 \\
		\textit{#3} & \textit{#4} \\
\end{tabular*}\vspace{-6pt}}

% New Command
\newcommand{\rehead}[2]{
\begin{tabular*}{7.40in}{l@{\extracolsep{\fill}}r}
        \textit{#1} & \textit{#2} \\
\end{tabular*}\vspace{-6pt}}
% New Command
\newcommand{\recomm}[2]{
\begin{tabular*}{7.40in}{l@{\extracolsep{\fill}}r}
        \textbf{#1} & \textit{#2} \\
\end{tabular*}\vspace{-6pt}}

\setlist[itemize]{leftmargin=0.7em}
%\setlist[itemize]{rightmargin=2.0em}
%-----------------------------------------------------------
%\frenchspacing

\begin{document}
%\frenchspacing
\begin{tabular*}{7.5in}{l@{\extracolsep{\fill}}r}
\textbf{\large Sonam K. Ghosh}  & 781-664-7151 (cell)\\
9 Chester St&  sonamg@bu.edu \\
Cambridge, MA  02140 & \faLinkedinSquare\hspace{0.1em}: linkedin.com/in/sonamghosh96  \faGithub\hspace{0.1em}: github.com/sonamghosh\\
\end{tabular*}
\\

\vspace{0.1in}



\resheading{Education}
\begin{itemize}
\item
	\ressubheading{Boston University}{Boston, MA}{Bachelor of Science in Electrical Engineering \& Minor in Physics, Cum Laude}{Feb. 2015 - May. 2018} 
	\begin{itemize}
	    \resitem{Major GPA: 3.68/4.0, Cum. GPA: 3.54/4.0}
		%\resitem{Relevant Coursework: Digital Signal Processing, Software Engineering, Quantum Mechanics I/II}
	\end{itemize}
\item	
	\ressubheading{National University of Singapore}{Singapore}{Undergraduate Exchange Student in Electrical/Computer Engineering \& Physics}{Jan. 2017 - May 2017}

\end{itemize}


\resheading{Skills}
\begin{description}
\item[Languages:]
Python, C/C{}\verb!++!, MATLAB, Java, Unix, \LaTeX, Javascript, Swift, C\#, ReactJS
\item[Software:]
Git, PyTorch, TensorFlow, SciPy, Pandas, Scikit, QuTip, OpenCV, Qiskit, Google Cloud, IBM Watson Cloud, Microsoft Azure Cloud
\end{description}



\resheading{Work Experience}
\begin{itemize}
\setlength{\leftmargini}{0.5em}
\item
	\ressubheading{MITRE}{Bedford, MA}{Machine Learning Signal Processing Engineer}{June 2018 - October 2018}
	\begin{itemize}
		%\resitem{Added support for mulitple platforms to the password escrow component of the %\emph{One Identity} Identity and Access Management suite.}
		%\resitem{Part of a team designing the Unix Privilege Management component of the \emph{One Identity} Identity and Access Management suite.}
		%\resitem{Implemented support for forwarding Unix domain sockets in OpenSSH and merged the changes upstream.}
		%\resitem{Continue to develop and maintain \emph{Sudo} and make regular releases.}
		\resitem{Developed \emph{Python} code to do data preprocessing, feature extraction, feature plotting, and classification of radio-frequency signals using machine learning and signal processing techniques.}
        \resitem{Utilized signal analysis techniques such as \emph{Empirical Mode Decomposition}, \emph{Hilbert analysis}, and \emph{Fourier-Bessel expansion analysis}.}
        \resitem{Utilized \emph{Machine Learning} models like \emph{Random Forest Classifiers}, \emph{Support Vector Machines}, \emph{Multi-Layer Perceptrons}, \emph{Naive Bayes}, and used \emph{t-Distributed Stochastic Neighbour Embedding} for feature plotting.}
	\end{itemize}
	%%%%
	\vspace{-3pt}
	\rehead{Signal Processing Engineer Intern}{May 2017 - May 2018}
	\begin{itemize}
		\resitem{Developed \emph{Python} classes and scripts related to Fault Generation in PNT applications.}
		\resitem{Developed \emph{C++} Physics \emph{Gravity model} for PNT application.}
	\end{itemize}
\item
	\ressubheading{Quantum Communication \& Measurement Lab at Boston University}{Boston, MA}{Undegraduate Quantum Computing/Optics Researcher}{September 2017 - May 2018}
	\begin{itemize}
	    \resitem{Defended \emph{Senior Honors Thesis} on demonstrating the potential of a \emph{linear-optical system} to implement \emph{quantum simulations of a molecule}.}
        \resitem{Constructed \& simulated the \emph{Hamiltonian}, calculating the \emph{eigenstates/eigenvalues}, and \emph{photon quantum walks} of a \emph{optical benzene topology} of connected directionally unbiased optical multiports using \emph{Python} and the \emph{QuTip} quantum mechanics software package}
	\end{itemize}
\item
	\ressubheading{Ultrafast Optics Lab at Boston University}{Boston, MA}{Research Assistant}{September 2015 - April 2016}
	\begin{itemize}
	    \resitem{Conducted \emph{spectral analysis} for custom built Photothermal Spectroscopy system}
	    \resitem{Learned the fundamentals of \emph{Photothermal Spectroscopy}, \emph{Fourier-Transform Infrared Spectroscopy}, and different types of \emph{Microscopy} }
	    \resitem{Designed \emph{Microscope Stage and Holder} setup, wrote protocols for various microscope setups, and made \emph{MATLAB} script for counting number of cells/circular objects in a sample}
	\end{itemize}

%\pagebreak

\end{itemize}



\resheading{Projects}
\begin{description}
\item[Wellesley Hacks 2018 Find StockAR --]Developed \emph{AR} app and website to identify logos for associated stocks with the ability to purchase shares, predict stock behaviour, and provide analyst advice. There was a \emph{Javascript} frontend for the website and \emph{ViroReact} backend for the AR and app. Developed the \emph{Deep Learning} and \emph{Data Science} stack which utilized finance APIs to form stock price datasets and a \emph{Dual Attention LSTM Recurrent Neural Network} model in \emph{PyTorch} to do \emph{time series} predicting. Finance APIs and the \emph{Plotly} package were used to make graphs and obtain stock metrics for analysis. The project \textbf{won} the \textbf{Kensho Case Study} and \textbf{Algolia search API} awards.

\item[Hack Harvard 2018 Music Boi --] Developed app which converts text into music. App was made into a website, app, and chrome extension using \emph{Javascript}. Developed the \emph{Deep learning backend} which called \emph{IBM Watson} Tone Analyzer API for \emph{sentiment analysis} to find the emotion of the text and a double stacked \emph{LSTM Recurrent Neural Network} model made in \emph{PyTorch} and trained on the \emph{Google Cloud} platform which produced music files to listen

\item[Perkins Hack 2018 Blind Gym --]  Developed proof of concept gym (Pathogym) for visually impaired individuals. Was in charge of creating a voice recognition component utilizing \emph{C\#} paired up with an \emph{Arduino} setup using \emph{C}.

\item[Make MIT 2018 CameraPiano --] Developed system where an individual with a camera can turn any surface into a playable piano. Utilized \emph{OpenCV} computer vision tools in \emph{Python} to detect fingers and play music notes. Future work involves scaling system and \emph{data analytics}

\end{description}


\resheading{Leadership Activities}
\begin{itemize}
    \item 	\ressubheading{Society of Asian Scientists \& Engineers}{Boston, MA}{Senior Advisor}{June 2018 - Present}
    \begin{itemize} 
        \item Advising and helping out with the 2018-2019 SASE E-Board with organization and logistics related to event planning, future events, and connections.
    \end{itemize} \vspace{-6pt}
    \rehead{Junior Representative, Secretary, Vice President}{May 2016 - May 2018}
    \begin{itemize} 
        \resitem{Created weekly powerpoints for E-Board meetings, managed the clubs email with both reading/responding, and organizing the Google Drive.}
        \resitem{Lead teams of two or three people to do various projects and tasks.}
        \resitem{Mentored and worked closely with freshman/sophomore representatives and corporate relations chair}
    \end{itemize}
\end{itemize}

\resheading{Interests \& Activities}
Society of Asian Scientists and Engineers (SASE), IEEE Eta Kappu Nu ECE Honor Member of Society, Machine Intelligence Community (MIC Club), Mentoring, Tutoring, Travelling the world and experiencing new cultures, Learning Japanese \& Korean



\begin{comment}
\pagebreak

\resheading{Publications}

\begin{description}
\item["Security-Enhanced Darwin: Porting SELinux to Mac OS X",]
\emph{Proceedings from the Third Annual Security Enhanced Linux Symposium} in Baltimore, MD, 2007.
\item["{\sc \bf UNIX} System Administration Handbook, Third Edition",]
contributing author.
\item["strlcpy and strlcat:  Consistent, Safe, String Copy and Concatenation",]
\emph{Proceedings from the USENIX Annual Technical Conference} in Monterey, CA, 1999.
\item["satool:  A System Administrators Cockpit",]
\emph{Proceedings from the USENIX LISA VII Conference (Large Installation Systems Administration)} in Monterey, CA, 1993.
\end{description}

\resheading{Patents}

\begin{description}
\item[Furlong, Wesley J., Schlossnagle, George, Miller, Todd.  2014.]
Method and system for adaptive delivery of digital messages.
U.S. Patent 8,782,184 filed September 20, 2013, and issued July 15, 2014.
\end{description}
\end{comment}



\end{document}
